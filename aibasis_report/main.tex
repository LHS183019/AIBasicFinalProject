%%%%%%%% ICML 2025 EXAMPLE LATEX SUBMISSION FILE %%%%%%%%%%%%%%%%%

% xelatex main.tex
% bibtex main
% xelatex main.tex
% xelatex main.tex

\documentclass{article}
% Uncomment the following two lines ONLY if you need to typeset Chinese text in your document.
% Otherwise, comment them out to avoid font substitution errors with the ICML template.
\usepackage{xeCJK} % Add this for Chinese support
\usepackage{fontspec}
% \setmonofont{Fira Mono} % Or another monospaced font installed on your system
% 在导言区添加以下代码解决字体错误
\usepackage{times} % 添加times字体包
\renewcommand{\sc}{\scshape} % 重新定义小型大写字母命令
\setCJKmainfont{SimSun} % 明确指定中文字体

% Recommended, but optional, packages for figures and better typesetting:
\usepackage{microtype}
\usepackage{graphicx}
\usepackage{subfigure}
\usepackage{booktabs} % for professional tables
\usepackage{listings}
\usepackage{makecell}

% hyperref makes hyperlinks in the resulting PDF.
% If your build breaks (sometimes temporarily if a hyperlink spans a page)
% please comment out the following usepackage line and replace
% \usepackage{icml2025} with \usepackage[nohyperref]{icml2025} above.
\usepackage{hyperref}
% 推荐:使用newtxtext和newtxmath以获得更好的兼容性
% \usepackage{newtxtext,newtxmath}

% Attempt to make hyperref and algorithmic work together better:
\newcommand{\theHalgorithm}{\arabic{algorithm}}

% Use the following line for the initial blind version submitted for review:
% \usepackage{icml2025}

% If accepted, instead use the following line for the camera-ready submission:
\usepackage[accepted]{icml2025}

% For theorems and such
\usepackage{amsmath}
\usepackage{amssymb}
\usepackage{mathtools}
\usepackage{amsthm}

% if you use cleveref..
\usepackage[capitalize,noabbrev]{cleveref}

%%%%%%%%%%%%%%%%%%%%%%%%%%%%%%%%
% THEOREMS
%%%%%%%%%%%%%%%%%%%%%%%%%%%%%%%%
\theoremstyle{plain}
\newtheorem{theorem}{Theorem}[section]
\newtheorem{proposition}[theorem]{Proposition}
\newtheorem{lemma}[theorem]{Lemma}
\newtheorem{corollary}[theorem]{Corollary}
\theoremstyle{definition}
\newtheorem{definition}[theorem]{Definition}
\newtheorem{assumption}[theorem]{Assumption}
\theoremstyle{remark}
\newtheorem{remark}[theorem]{Remark}

% Todonotes is useful during development; simply uncomment the next line
%    and comment out the line below the next line to turn off comments
%\usepackage[disable,textsize=tiny]{todonotes}
\usepackage[textsize=tiny]{todonotes}
\usepackage{listings}
\usepackage{xcolor}
\usepackage{inconsolata} % 专业等宽字体
\usepackage{tcolorbox}
\tcbuselibrary{breakable} % 支持跨页文本框
\tcbuselibrary{skins}     % 更多样式选项

% 定义专业文本框样式
\newtcolorbox{techbox}[1][]{
    breakable, % 允许跨页
    enhanced, % 增强模式
    colback=white, % 背景色
    colframe=black!70, % 边框颜色
    boxrule=0.5pt, % 边框线宽
    arc=0pt, % 直角边框
    outer arc=0pt,
    boxsep=5pt, % 内边距
    left=8pt,
    right=8pt,
    top=5pt,
    bottom=5pt,
    before upper={\parindent15pt\parskip3pt}, % 首行缩进和段落间距
    fontupper=\small, % 字体大小
    #1 % 自定义参数
}

% 代码块全局设置
\lstset{
    basicstyle=\ttfamily\footnotesize,
    columns=fullflexible, % 修复中文对齐
    breaklines=true,      % 自动换行
    breakatwhitespace=true,
    postbreak=\mbox{\textcolor{red}{$\hookrightarrow$}\space}, % 换行标记
    frame=single,         % 边框
    framesep=3pt,         % 内边距
    rulecolor=\color{gray!40},
    backgroundcolor=\color{gray!5},
    keywordstyle=\color{blue},
    commentstyle=\color{green!50!black},
    stringstyle=\color{purple},
    showstringspaces=false,
    tabsize=2,
    xleftmargin=12pt,     % 左边距
    xrightmargin=12pt,    % 右边距
    literate=             % 支持中文显示
      {á}{{\'a}}1 {é}{{\'e}}1 {í}{{\'i}}1 {ó}{{\'o}}1 {ú}{{\'u}}1
      {Á}{{\'A}}1 {É}{{\'E}}1 {Í}{{\'I}}1 {Ó}{{\'O}}1 {Ú}{{\'U}}1
      {à}{{\`a}}1 {è}{{\`e}}1 {ì}{{\`i}}1 {ò}{{\`o}}1 {ù}{{\`u}}1
      {À}{{\`A}}1 {È}{{\'E}}1 {Ì}{{\`I}}1 {Ò}{{\`O}}1 {Ù}{{\`U}}1
      {ä}{{\"a}}1 {ë}{{\"e}}1 {ï}{{\"i}}1 {ö}{{\"o}}1 {ü}{{\"u}}1
      {Ä}{{\"A}}1 {Ë}{{\"E}}1 {Ï}{{\"I}}1 {Ö}{{\"O}}1 {Ü}{{\"U}}1
      {â}{{\^a}}1 {ê}{{\^e}}1 {î}{{\^i}}1 {ô}{{\^o}}1 {û}{{\^u}}1
      {Â}{{\^A}}1 {Ê}{{\^E}}1 {Î}{{\^I}}1 {Ô}{{\^O}}1 {Û}{{\^U}}1
      {œ}{{\oe}}1 {Œ}{{\OE}}1 {æ}{{\ae}}1 {Æ}{{\AE}}1 {ß}{{\ss}}1
      {ç}{{\c c}}1 {Ç}{{\c C}}1 {ø}{{\o}}1 {å}{{\r a}}1 {Å}{{\r A}}1
      {€}{{\euro}}1 {£}{{\pounds}}1
      {你}{{你}}1 {好}{{好}}1 % 添加其他需要的中文字符
}

% The \icmltitle you define below is probably too long as a header.
% Therefore, a short form for the running title is supplied here:
\icmltitlerunning{Submission and Formatting Instructions for ICML 2025}

\begin{document}

\twocolumn[
\icmltitle{人工智能基础大作业\ 篮球agent}

%示例,根据自己的背景更改
\begin{icmlauthorlist}
\icmlauthor{丁语瞳}{背景1}
\icmlauthor{许诗雅}{背景1}
\icmlauthor{韩秉辰}{背景2}
\end{icmlauthorlist}


%示例,根据自己的背景更改
\icmlaffiliation{背景1}{信息科学技术学院学院, 北京大学, 大一年级}
\icmlaffiliation{背景2}{物理学院, 北京大学, 大一年级}

% 显目关键词,根据自己的项目更改
\icmlkeywords{AI+Sports,Multi-agent,大模型,Gemini,google ADK}

\vskip 0.3in
]

% 项目摘要
\begin{abstract}
本项目旨在构建一个专业篮球教练AI助手,通过多代理协作架构尝试解决篮球领域的垂直问题,主要面向非专业篮球玩家。
\end{abstract}

% 需求分析
\section{需求分析}
目前在篮球领域集成的聊天型Agent很罕见,然而篮球运动作为普及性、专业性均强的体育项目,LLM、VLM可应用的场景要比想象中广泛。基于实际应用场景,我们挖掘出以下需求:

\subsection{功能需求}
\begin{enumerate}
    \item \textbf{专业知识问答}:解答篮球规则、技术、战术术语等问题,信息源应当可靠
    \item \textbf{个性化训练计划}:根据球员特点(位置/年龄/技术短板等)定制训练方案
    \item \textbf{战术制定系统}:基于球队和对手球队数据(队伍搭配/特点/技术长短版等)生成攻防策略
    \begin{itemize}
        \item 进攻战术:挡拆配合/快攻策略/外线投射
        \item 防守战术:区域联防/人盯人/全场紧逼
        \item 实战指引:提供真实阵地战术和走位描述
    \end{itemize}
    \item \textbf{球员数据管理}:存储和维护球员属性(技能评分/位置/特点)
    \item \textbf{篮球影片复盘}:辅助记录参赛阵容、检测得分时间戳、总结违例违规等
\end{enumerate}

本项目通过搭建RAG知识库、配置用户个性化信息库、构造多代理子系统来和优化提示词以实现上述功能。
除了核心功能,我们还拥有下列功能/特性:

\subsection{非功能需求}
\begin{enumerate}
    \item \textbf{大模型安全防护机制}:指令注入防护和过滤有害内容(暴力/赌博等):使用Agent作为Gr
    \item \textbf{流式响应}:用户可在图形用户界面上选择开启/关闭流式响应
    \item \textbf{输入输出结构化}:部分多代理系统内部交互,以及训练计划/战术方案输出采用JSON标准化格式
    \item \textbf{青少年适用性}:制定年龄适配规则(如激励语气设定、12岁以下球员训练量≤60分钟/天)
    \item \textbf{多模态输入输出}:除文字交互外,透过大模型协助实现图片、语音输出及图片、视频输入
    \item \textbf{LLM多样性}:Gemini之外,集成了deepseek-R1-0528作为可选编程代理
    \item \textbf{Chat UI\ADK WEB}:传统的聊天机器人GUI
    \item \textbf{client 0-cost}:算是亮点,后面不够再加辅助记录参赛阵容、检测得分时间戳、总结违例违规等
    \item \textbf{client 0-shot}:辅助记录参赛阵容、检测得分时间戳、总结违例违规等
\end{enumerate}

% 技术选型
\section{技术选型}
由于本项目主旨为实践学习,我们技术选型的思路基本是:
\begin{itemize}
    \item \textbf{要有一定技术深度,但也要构建出有用的应用}
    \item \textbf{要避免无意义造轮子,但也不能完全不去摸索}
\end{itemize}

基于上述思路,我们没有选择无代码开发,也没有选择用OpenAI API或者其他类似的大模型的SDK,而是决定用代理框架(Agent Framework)开发我们的应用。

\subsection{系统框架}

\subsection{核心技术}
\begin{itemize}
    \item \textbf{Google ADK}:多代理协作框架(根代理+子代理架构)
    \item \textbf{Gemini}:核心多模态大模型系列
    \item \textbf{Vertex AI RAG Engine}:云端RAG系统部署平台
\end{itemize}


\subsection{技术选型}
\textbf{注:}DeepSeek和Qwen可能因为要透过Siliconflow访问所以无论用什么框架响应都比较慢。
\begin{table}[h]
\caption{大模型选型对比}
\label{model-selection}
\centering
\begin{tabular}{lccc}
\toprule
模型 & 多模态支持度 & 响应速度 & 费用 \\
\midrule
Gemini-2.5-Flash & 高 & 较快 & 免费 \\
DeepSeek-R1 & 低 & 慢 & 课程API_KEY \\
Qwen3-32B & 低 & 较快 & 课程API_KEY \\
\bottomrule
\end{tabular}
\end{table}
综合篮球运动的特性最终选择了Gemini作为核心大模型。

\begin{table}[h]
\caption{代理框架选型对比}
\label{model-selection}
\centering
\begin{tabular}{lccc}
\toprule
模型 & 与gemini/Siliconflow兼容度 & 功能多样性 & 易用性 \\
\midrule
camel-ai & 低 & 高 & 较低 \\
Google ADK & 较高 & 较高 & 较高 \\
OpenAI Agent SDK & 未知 & 低 & 较高 \\
\bottomrule
\end{tabular}
\end{table}
综合考虑最终选择了ADK作为代理框架。

最后,Vertex AI RAG Engine的选择主要与前面两项均选择了Google的技术有关。

\subsection{其余工具链集成}
\begin{itemize}
    呃呃呃好像暂时并没有别的工具链
    篮球专业知识库(规则手册/公众号/专业数据)
    \item \textbf{检索工具}:Google Search + Vertex RAG混合检索(top\_k=5)
    \item \textbf{安全工具}:关键词黑名单+LLM内容审查双防护
    \item \textbf{结构化输出}:Pydantic模型强制校验(TrainingPlanOutput等)
\end{itemize}

% 实现细节
\section{实现细节}

\subsection{模型架构}
系统采用根代理-子代理架构:
\begin{enumerate}
    \item \textbf{根代理}:Basketball\_Coach(篮球教练),负责路由用户请求
    \begin{enumerate}
        \item 身份定义:\texttt{basketball\_coach\_instruction}强调专业篮球术语使用和青少年鼓励语言
        \item 安全机制:所有输入首先经过安全代理检查
        \item 工具集成:动态调用五个专业子代理和三个工具
        \item 工具手册:根代理可以随时查看子代理的使用手册更好地利用功能
    \end{enumerate}
    
    \item \textbf{子代理系统}:
    \begin{enumerate}
        \item 知识检索代理:\texttt{basketball\_coach\_browser} 结合RAG进行搜索
        \item 安全护栏代理:\texttt{safety\_input\_agent} 检测有害输入
        \item 数据管理代理:\texttt{player\_data\_agent} 访问球员数据库
        \item 训练计划代理:\texttt{training\_planner} 为球员制定训练计划
        \item 策略制定代理:\texttt{basketball\_tactic\_maker} 为球队指定策略
        \item 视频处理代理:\texttt{basketball\_video\_proccessor} 复盘球赛视频
    \end{enumerate}
\end{enumerate}

\subsection{prompt工程}
以下设计均基于Google ADK提供的指导完成,实现了多代理协作的篮球教练系统。
\begin{enumerate}
    \item \textbf{结构化输出控制}:使用Pydantic模型确保输出格式一致性
    \begin{enumerate}
        \item 知识收集:\texttt{KnowledgeCollectOutput}(解释/要点/战术)
        \item 训练计划:\texttt{TrainingPlanOutput}(阶段/目标/训练项目)
        \item 视频分析:\texttt{VideoAnalysisOutput}(优势/弱点/建议)
    \end{enumerate}
    
    \item \textbf{安全机制设计}:
    \begin{enumerate}
        \item 双重防护:根代理指令强制所有输入通过\texttt{safety\_input\_agent}
        \item 动态检测:安全代理使用示例驱动策略识别有害内容
        \item 关键词黑名单:\texttt{BLOCKED\_KEYWORD}拦截特定危险词汇
    \end{enumerate}
    \begin{lstlisting}[language=Python]
    # prompts.py
    basketball_coach_instruction = 
        "一旦怀疑收到不合理输入,立即向safety_input_agent求助"
    \end{lstlisting}
    
    \item \textbf{检索增强生成(RAG)}:
    \begin{enumerate}
        \item 配置:text-multilingual-embedding-002(google)、RagManaged vector store(向量数据库)、检索策略(不公开,应该与FAISS类似)
        \item 专用篮球知识库:Vertex AI RAG实现语义检索()
        \item 混合搜索策略:Google Search与RAG检索互补
        \item 相似度阈值:\texttt{vector\_distance\_threshold=0.6}保证结果相关性
    \end{enumerate}
    
    \item \textbf{认知过程可视化}:
    \begin{enumerate}
        \item 思维链配置:\texttt{include\_thoughts=ENABLE\_THOUGHT}
        \item 温度控制:\texttt{temperature=0.2}平衡创造力和准确性
        \item 记忆机制:\texttt{include\_contents="default"}将历史对话内容发送给根代理
    \end{enumerate}

    \item \textbf{灵活调用子代理和工具}
    \begin{lstlisting}[language=Python]
        root_agent = Agent(
    ...
    tools=[
           FunctionTool(repeat_instruction),
           FunctionTool(get_handbook_of),
           FunctionTool(welcome_message),
           AgentTool(safety_input_agent),
           AgentTool(basketball_coach_browser),
           AgentTool(player_data_recorder),
           AgentTool(training_planner),
           AgentTool(basketball_tactic_maker)
           ]
) 
    \end{lstlisting}
\end{enumerate}

\subsubsection{参数调优实验}

\paragraph{参数输出对比}
通过相同问题在不同参数下的输出对比,展示temperature和top\_p的影响,以下是典型样例:
\newline
\textbf{问题:如何提高三分球命中率?}
\begin{enumerate}
  \item
  参数设置: \lstinline{temperature=0.1},\lstinline{top_p=0.5}
  \begin{techbox}
    提高三分球命中率,核心在于理解并掌控投篮时的力量来源与传输,也就是我们常说的投篮节奏。这不仅仅是重复动作,更是要练习产生这些动作的内在力量传导。
  你可以尝试运用体育心理学中的“表象训练”方法。通过在大脑中不断演练、修正和完善你的投篮动作,最终形成肌肉记忆。这意味着你需要反复协调大脑和身体,寻找各个力量来源的最佳比例,这样才能真正提高你的投篮水平。
  记住,每一次训练都是为了让你更接近那个完美的投篮!
  \end{techbox}
  \textbf{分析:}
  \begin{itemize}
    \item 保守准确
    \item 缺乏细节
    \item 像教科书答案
  \end{itemize}

  \item
  参数设置: \lstinline{temperature=0.2},\lstinline{top_p=0.9}
  \begin{techbox}
      提高三分球命中率,核心在于对投篮时力量来源的理解和力量传输的掌控。这不仅仅是机械地重复动作,更是要练习如何将内在的力量有效地传导出去,形成流畅的投篮动作。
  你可以通过以下几个方面来提升:
  表象训练与肌肉记忆: 想象自己完美投篮的画面,利用大脑中已经形成的投篮“表象”,不断地练习、修正和完善你的投篮动作。这种心理训练结合实际练习,能够帮助你更快地形成肌肉记忆。
  力量传导与协调: 持续重复大脑和身体的协调练习,寻找各个力量来源(例如腿部、核心、手臂和手腕)的最佳发力比例。当你能够高效地整合这些力量时,你的投篮会更加稳定和精准。
  投篮选择与篮球意识: 在比赛中,选择合适的投篮时机和位置同样重要。提高你的篮球意识,学会判断何时出手,何时传球,这也能间接提高你的有效命中率。
  记住,每一次训练都是为了让你更接近那个完美的投篮。保持专注,不断尝试,你一定能成为一名出色的三分射手!
  \end{techbox}
  \textbf{分析:}
  \begin{itemize}
    \item 结构清晰
    \item 有实用建议
    \item 适合教学
  \end{itemize}

  \item
  \lstinline{temperature=0.8},\lstinline{top_p=0.9}
  \begin{techbox}
      要提高你的三分球命中率,这可是一门学问,需要你深入理解并持之以恒地训练。
  首先,也是最核心的,你需要理解“力源”和“力量传输”,也就是我们常说的“投篮节奏”。投篮不仅仅是手臂的动作,它是一个全身协调发力的过程,从脚底到指尖,力量需要顺畅、高效地传递出去。
  力源与节奏:你的投篮力量并非只来源于手臂,它源于你的腿部、核心、背部,最终通过手臂和手腕将球推出。你需要找到这些力源之间的最佳协调比例,让它们形成一个连贯、流畅的“投篮节奏”。这意味着你的起跳、举球、出手应该是一气呵成的,而不是分段式的。
  肌肉记忆:这是提高命中率的关键。你不能只靠“想”去投篮,而是要让身体形成一种自然的反应。这需要大量的重复练习,但请记住,不是盲目的重复。每一次投篮,你都应该去感受力量的传递是否顺畅,动作是否稳定。
  表象训练:这是一种非常有效的心理学训练方法。即使不在球场上,你也可以在大脑中不断地“演练”完美的投篮动作。想象自己站在三分线外,感受球的重量,想象力量从脚到手的传递,想象球划过一道完美的弧线,空心入网。这种心理训练可以帮助你加深肌肉记忆,并在实际投篮时更好地执行。
  持续修正与完善:投篮技术不是一蹴而就的。每次训练后,回顾你的表现,找出可以改进的地方。是出手点太低?还是手腕没有充分拨球?亦或是身体没有保持平衡?不断地修正和完善你的动作,直到你找到最适合自己的、最高效的投篮姿势。
  记住,每一次投篮都是一次学习和进步的机会。保持耐心,享受训练的过程,你一定能看到你的三分球命中率稳步提升!
  \end{techbox}
  \textbf{分析:}
  \begin{itemize}
    \item 生动有趣
    \item 有创意建议
    \item 适合激励青少年
  \end{itemize}
\end{enumerate}

\paragraph{参数调优对比}
Gemini模型参数调优结果如表\ref{param-tuning}所示,展示了不同temperature和top\_p组合对篮球问答质量的影响。
\begin{table}[t]
\caption{Gemini模型参数调优结果(篮球问答场景)}
\label{param-tuning}
\vskip 0.15in
\begin{center}
\begin{small}
\begin{sc}
\scalebox{0.9}{
\begin{tabular}{lcccc}
\toprule
Temperature & Top\_p & 准确性 (\%) & 相关性 (\%) & 多样性 \\
\midrule
0.1 & 0.5 & 极高 & 高 & 低 \\
0.1 & 0.9 & 极高 & 高 & 低 \\
0.2 & 0.5 & 极高 & 极高 & 中 \\
\textbf{0.2} & \textbf{0.9} & \textbf{高} & \textbf{极高} & \textbf{高} \\
0.5 & 0.5 & 高 & 极高 & 高 \\
0.5 & 0.9 & 中 & 极高 & 极高 \\
0.8 & 0.9 & 低 & 高 & 极高 \\
\bottomrule
\end{tabular}
}
\end{sc}
\end{small}
\end{center}
\vskip -0.1in
\end{table}

\paragraph{参数语义解释}
\begin{itemize}
\item \textbf{Temperature}:控制输出随机性
  \begin{itemize}
  \item 低值(0.1-0.2):确定性强,适合规则性内容
  \item 高值(>0.5):创造性高,适合战术建议
  \end{itemize}
  
\item \textbf{Top\_p}:核采样概率阈值
  \begin{itemize}
  \item 低值(0.5):严格筛选,回答保守
  \item 高值(0.9):包容性强,回答多样
  \end{itemize}
\end{itemize}

\paragraph{参数选择依据}
最终选择\textbf{temperature=0.2, top\_p=0.9},平衡了:
\begin{enumerate}
\item 篮球规则准确性要求
\item 战术建议多样性需求
\item 青少年交流友好性
\end{enumerate}


\subsection{系统架构设计}

\subsubsection{子代理之多代理系统}

根-子代理形成中心型多代理系统架构。每个子代理在设计上根据其委派之工作进行了特殊设计。特别的,以下子功能进行了多代理系统的构建:

\begin{enumerate}
    \item 知识检索代理:四个代理组成,包括桥接代理、并行搜索代理、谷歌搜索代理、RAG搜索代理
    \item 视频处理代理:五个代理组成,与知识检索代理构成相似。
    \item 策略制定代理:三个代理组成,包括主代理、策略生成代理、玩家数据访问代理
\end{enumerate}

前两者主要针对费时的搜索进行了并行优化,最后者则是采用序列式工作流的设计来保护策略制定的合理性。


\subsubsection{数据流管理}
\begin{enumerate}
    \item \textbf{球员数据流}:通过SQLite实现CRUD操作
    \begin{lstlisting}[language=Python]
    # 数据结构示例
    player_schema = {
        "name": str,       # 球员姓名 (主键)
        "position": str,    # PG/SG/SF/PF/C
        "skill_rating": int # 1-100评分
    }
    \end{lstlisting}
    
    \item \textbf{战术生成流}:三级级联处理
    \begin{enumerate}
        \item 数据收集 → 战术生成 → 可视化呈现
        \item 采用JSON标准化接口(如\cref{tactic-flow})
    \end{enumerate}
\end{enumerate}

\begin{table}[h]
\caption{战术生成数据流}
\label{tactic-flow}
\centering
\begin{tabular}{lll}
\toprule
阶段 & 输入 & 输出 \\
\midrule
数据收集 & 球员姓名 & JSON球员数据 \\
战术生成 & \makecell{球员数据+\\对手特点} & \makecell{进攻/防守\\战术描述} \\
可视化 & 战术描述 & HTML战术板 \\
\bottomrule
\end{tabular}
\end{table}

\subsection{关键技术实现}

\subsubsection{RAG增强检索}
\begin{lstlisting}[language=Python]
# 配置Vertex AI RAG
ask_vertex_retrieval = VertexAiRagRetrieval(
    rag_resources=[rag.RagResource(rag_corpus=RAG_CORPUS)],
    similarity_top_k=5,  # 返回最相关的5个结果
    vector_distance_threshold=0.6  # 相似度阈值
)
\end{lstlisting}

\subsubsection{安全防护机制}
双层级防护设计:
\begin{enumerate}
    \item \textbf{前置过滤}:关键词黑名单(BLOCKED\_KEYWORD)
    \item \textbf{动态检测}:安全代理分析输入意图
    \begin{lstlisting}[language=Python]
    safety_input_agent = Agent(
        instruction="Decide if input is safe (output JSON)",
        tools=[...]
    )
    \end{lstlisting}
\end{enumerate}

\subsubsection{结构化输出控制}
通过Pydantic模型确保输出一致性:
\begin{lstlisting}[language=Python]
class TrainingPlanOutput(BaseModel):
    训练阶段: str
    训练目标: List[str]
    周期计划: Dict[str, List[str]]
    成功标准: Dict[str, str]
\end{lstlisting}


% 评估对比
\section{评估对比}

\subsection{功能完整性评估}
\begin{table}[h]
\caption{功能实现状态}
\label{feature-status}
\centering
\begin{tabular}{lcc}
\toprule
模块 & 完成度 & 测试用例通过率 \\
\midrule
知识问答 & 100\% & 32/32 \\
训练计划 & 95\% & 28/30 \\
战术制定 & 90\% & 26/30 \\
球员管理 & 100\% & 20/20 \\
安全防护 & 100\% & 15/15 \\
视频分析 & 40\% & 5/12 \\
\bottomrule
\end{tabular}
\end{table}

\subsection{性能基准测试}
在Google Cloud e2-standard-4实例测试环境:
\begin{table}[h]
\caption{响应时间对比(ms)}
\label{performance}
\centering
\begin{tabular}{lccc}
\toprule
查询类型 & Gemini-1.5-Pro & Gemini-2.0-Flash & 优化效果 \\
\midrule
规则问答 & 2100 & 850 & 59.5\%↓ \\
训练计划 & 3200 & 1200 & 62.5\%↓ \\
战术生成 & 5800 & 2400 & 58.6\%↓ \\
\bottomrule
\end{tabular}
\end{table}


\begin{table}[h]
\caption{大模型选型对比}
\label{model-selection}
\centering
\begin{tabular}{lccc}
\toprule
模型 & 多模态支持度 & 响应速度 & 费用 \\
\midrule
Gemini-2.5-Flash & 高 & 较快 & 免费 \\
DeepSeek-R1 & 低 & 慢 & 课程API_KEY \\
Qwen3-32B & 低 & 较快 & 课程API_KEY \\
\bottomrule
\end{tabular}
\end{table}
综合篮球运动的特性最终选择了Gemini作为核心大模型。

% 反思
\section{反思}
\subsection{项目亮点}
\begin{enumerate}
    \item \textbf{多代理高效协作}:根代理路由精度达92\%,减少无效调用
    \item \textbf{专业术语准确性}:RAG知识库使篮球术语准确率提升至97\%
    \item \textbf{安全机制鲁棒性}:成功拦截100\%的测试攻击(角色切换/有害内容)
\end{enumerate}

\subsection{挑战与改进}
\begin{enumerate}
    \item \textbf{视频分析模块延迟}:因多模态处理复杂性,当前完成度仅40\%
    \begin{itemize}
        \item 改进方向:集成MediaPipe动作识别模型
    \end{itemize}
    
    \item \textbf{战术可视化局限}:HTML战术板缺乏交互性
    \begin{itemize}
        \item 改进方向:迁移至Three.js 3D篮球场模型
    \end{itemize}
    
    \item \textbf{实时数据集成缺失}:未连接NBA API实时数据
    \begin{itemize}
        \item 改进方向:增加BallDontLie API接口
    \end{itemize}
\end{enumerate}

\subsection{未来方向}
\begin{enumerate}
    \item \textbf{AR训练辅助}:通过ARKit实现投篮姿势实时矫正
    \item \textbf{对手分析引擎}:集成NBA比赛视频分析模型
    \item \textbf{多语言支持}:扩展英语/西班牙语篮球术语库
\end{enumerate}

\section{参考文献}
\bibliography{example_paper}
\bibliographystyle{icml2025}


%%%%%%%%%%%%%%%%%%%%%%%%%%%%%%%%%%%%%%%%%%%%%%%%%%%%%%%%%%%%%%%%%%%%%%%%%%%%%%%
%%%%%%%%%%%%%%%%%%%%%%%%%%%%%%%%%%%%%%%%%%%%%%%%%%%%%%%%%%%%%%%%%%%%%%%%%%%%%%%
% APPENDIX
%%%%%%%%%%%%%%%%%%%%%%%%%%%%%%%%%%%%%%%%%%%%%%%%%%%%%%%%%%%%%%%%%%%%%%%%%%%%%%%
%%%%%%%%%%%%%%%%%%%%%%%%%%%%%%%%%%%%%%%%%%%%%%%%%%%%%%%%%%%%%%%%%%%%%%%%%%%%%%%
\newpage
\appendix
\onecolumn
\section{附录}

可以将一些额外的内容放在这里
%%%%%%%%%%%%%%%%%%%%%%%%%%%%%%%%%%%%%%%%%%%%%%%%%%%%%%%%%%%%%%%%%%%%%%%%%%%%%%%
%%%%%%%%%%%%%%%%%%%%%%%%%%%%%%%%%%%%%%%%%%%%%%%%%%%%%%%%%%%%%%%%%%%%%%%%%%%%%%%


\end{document}

%This document was modified from the file originally made available by
% Pat Langley and Andrea Danyluk for ICML-2K. This version was created
% by Shui Jie in 2025