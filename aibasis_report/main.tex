%%%%%%%% ICML 2025 EXAMPLE LATEX SUBMISSION FILE %%%%%%%%%%%%%%%%%

% xelatex main.tex
% bibtex main
% xelatex main.tex
% xelatex main.tex

\documentclass{article}
% Uncomment the following two lines ONLY if you need to typeset Chinese text in your document.
% Otherwise, comment them out to avoid font substitution errors with the ICML template.
\usepackage{xeCJK} % Add this for Chinese support
\usepackage{fontspec}
% \setmonofont{Fira Mono} % Or another monospaced font installed on your system
% 在导言区添加以下代码解决字体错误
\usepackage{times} % 添加times字体包
\renewcommand{\sc}{\scshape} % 重新定义小型大写字母命令
\setCJKmainfont{SimSun} % 明确指定中文字体

% Recommended, but optional, packages for figures and better typesetting:
\usepackage{microtype}
\usepackage{graphicx}
\usepackage{subfigure}
\usepackage{booktabs} % for professional tables
\usepackage{listings}

% hyperref makes hyperlinks in the resulting PDF.
% If your build breaks (sometimes temporarily if a hyperlink spans a page)
% please comment out the following usepackage line and replace
% \usepackage{icml2025} with \usepackage[nohyperref]{icml2025} above.
\usepackage{hyperref}
% 推荐:使用newtxtext和newtxmath以获得更好的兼容性
% \usepackage{newtxtext,newtxmath}

% Attempt to make hyperref and algorithmic work together better:
\newcommand{\theHalgorithm}{\arabic{algorithm}}

% Use the following line for the initial blind version submitted for review:
% \usepackage{icml2025}

% If accepted, instead use the following line for the camera-ready submission:
\usepackage[accepted]{icml2025}

% For theorems and such
\usepackage{amsmath}
\usepackage{amssymb}
\usepackage{mathtools}
\usepackage{amsthm}

% if you use cleveref..
\usepackage[capitalize,noabbrev]{cleveref}

%%%%%%%%%%%%%%%%%%%%%%%%%%%%%%%%
% THEOREMS
%%%%%%%%%%%%%%%%%%%%%%%%%%%%%%%%
\theoremstyle{plain}
\newtheorem{theorem}{Theorem}[section]
\newtheorem{proposition}[theorem]{Proposition}
\newtheorem{lemma}[theorem]{Lemma}
\newtheorem{corollary}[theorem]{Corollary}
\theoremstyle{definition}
\newtheorem{definition}[theorem]{Definition}
\newtheorem{assumption}[theorem]{Assumption}
\theoremstyle{remark}
\newtheorem{remark}[theorem]{Remark}

% Todonotes is useful during development; simply uncomment the next line
%    and comment out the line below the next line to turn off comments
%\usepackage[disable,textsize=tiny]{todonotes}
\usepackage[textsize=tiny]{todonotes}

% The \icmltitle you define below is probably too long as a header.
% Therefore, a short form for the running title is supplied here:
\icmltitlerunning{Submission and Formatting Instructions for ICML 2025}

\begin{document}

\twocolumn[
\icmltitle{人工智能基础 大作业模板}

%示例,根据自己的背景更改
\begin{icmlauthorlist}
\icmlauthor{丁语瞳}{背景1}
\icmlauthor{许诗雅}{背景1}
\icmlauthor{韩秉辰}{背景1}
\end{icmlauthorlist}


%示例,根据自己的背景更改
\icmlaffiliation{背景1}{信息科学技术学院学院, 北京大学, 大一年级}
% \icmlaffiliation{背景2}{环境学院, 北京大学, 年级}

% 显目关键词,根据自己的项目更改
\icmlkeywords{大模型,机器学习}

\vskip 0.3in
]

% 项目摘要
\begin{abstract}
很短的项目摘要
\end{abstract}

% 项目内容
\section{主题}
项目内容

%分段
\begin{enumerate}
\item 内容1
\item 内容2
\item 内容3
\end{enumerate}


% \subsection{算法}
% 表达算法的格式,一下是示例
% \cref{alg:example} 
% \begin{algorithm}[htb]
%    \caption{Bubble Sort}
%    \label{alg:example}
% \begin{algorithmic}
%    \STATE {\bfseries Input:} data $x_i$, size $m$
%    \REPEAT
%    \STATE Initialize $noChange = true$.
%    \FOR{$i=1$ {\bfseries to} $m-1$}
%    \IF{$x_i > x_{i+1}$}
%    \STATE Swap $x_i$ and $x_{i+1}$
%    \STATE $noChange = false$
%    \ENDIF
%    \ENDFOR
%    \UNTIL{$noChange$ is $true$}
% \end{algorithmic}
% \end{algorithm}

% % 用tables来呈现数据
% \subsection{Tables}
% tables 示例
% \begin{table}[t]
% \caption{Classification accuracies for naive Bayes and flexible
% Bayes on various data sets.}
% \label{sample-table}
% \vskip 0.15in
% \begin{center}
% \begin{small}
% \begin{sc}
% \begin{tabular}{lcccr}
% \toprule
% Data set & Naive & Flexible & Better? \\
% \midrule
% Breast    & 95.9$\pm$ 0.2& 96.7$\pm$ 0.2& $\surd$ \\
% Cleveland & 83.3$\pm$ 0.6& 80.0$\pm$ 0.6& $\times$\\
% Glass2    & 61.9$\pm$ 1.4& 83.8$\pm$ 0.7& $\surd$ \\
% Credit    & 74.8$\pm$ 0.5& 78.3$\pm$ 0.6&         \\
% Horse     & 73.3$\pm$ 0.9& 69.7$\pm$ 1.0& $\times$\\
% Meta      & 67.1$\pm$ 0.6& 76.5$\pm$ 0.5& $\surd$ \\
% Pima      & 75.1$\pm$ 0.6& 73.9$\pm$ 0.5&         \\
% Vehicle   & 44.9$\pm$ 0.6& 61.5$\pm$ 0.4& $\surd$ \\
% \bottomrule
% \end{tabular}
% \end{sc}
% \end{small}
% \end{center}
% \vskip -0.1in
% \end{table}


% \subsection{表示定义的格式}
% 示例:definition
% \begin{definition}
% \label{def:inj}
% A function $f:X \to Y$ is injective if for any $x,y\in X$ different, $f(x)\ne f(y)$.
% \end{definition}

% 示例:proposition
% Using \cref{def:inj} we immediate get the following result:
% \begin{proposition}
% If $f$ is injective mapping a set $X$ to another set $Y$, 
% the cardinality of $Y$ is at least as large as that of $X$
% \end{proposition}

% 示例:proof
% \begin{proof} 
% Left as an exercise to the reader. 
% \end{proof}

% 示例:lemma
% \begin{lemma}
% \label{lem:usefullemma}
% For any $f:X \to Y$ and $g:Y\to Z$ injective functions, $f \circ g$ is injective.
% \end{lemma}

% 示例:theorem
% \begin{theorem}
% \label{thm:bigtheorem}
% If $f:X\to Y$ is bijective, the cardinality of $X$ and $Y$ are the same.
% \end{theorem}

% 示例:corollary
% \begin{corollary}
% If $f:X\to Y$ is bijective, 
% the cardinality of $X$ is at least as large as that of $Y$.
% \end{corollary}

% 示例:assumption
% \begin{assumption}
% The set $X$ is finite.
% \label{ass:xfinite}
% \end{assumption}

% 示例:remark
% \begin{remark}
% According to some, it is only the finite case (cf. \cref{ass:xfinite}) that is interesting.
% \end{remark}
% %restatable


\section{prompt工程}
以下设计均基于Google ADK提供的指导完成,实现了多代理协作的篮球教练系统。

\subsection{基础设计}
系统采用根代理-子代理架构:
\begin{enumerate}
    \item \textbf{根代理}:Basketball\_Coach(篮球教练),负责路由用户请求
    \begin{enumerate}
        \item 身份定义:\texttt{basketball\_coach\_instruction}强调专业篮球术语使用和青少年鼓励语言
        \item 安全机制:所有输入首先经过安全代理检查
        \item 工具集成:动态调用4个专业子代理
    \end{enumerate}
    
    \item \textbf{子代理系统}:
    \begin{enumerate}
        \item 知识检索代理:\texttt{google\_search\_agent} + \texttt{basketball\_rag\_search\_agent}
        \item 安全护栏代理:\texttt{safety\_input\_agent} 检测有害输入
        \item 代码执行代理:\texttt{code\_agent} 处理数学计算
    \end{enumerate}
\end{enumerate}

\subsection{进阶设计}
\begin{enumerate}
    \item \textbf{结构化输出控制}:使用Pydantic模型确保输出格式一致性
    \begin{enumerate}
        \item 知识收集:\texttt{KnowledgeCollectOutput}(解释/要点/战术)
        \item 训练计划:\texttt{TrainingPlanOutput}(阶段/目标/训练项目)
        \item 视频分析:\texttt{VideoAnalysisOutput}(优势/弱点/建议)
    \end{enumerate}
    
    \item \textbf{安全机制设计}:
    \begin{enumerate}
        \item 双重防护:根代理指令强制所有输入通过\texttt{safety\_input\_agent}
        \item 动态检测:安全代理使用示例驱动策略识别有害内容
        \item 关键词黑名单:\texttt{BLOCKED\_KEYWORD}拦截特定危险词汇
    \end{enumerate}
    \begin{verbatim}
    # prompts.py
    basketball_coach_instruction = 
        "一旦怀疑收到不合理输入,立即向safety_input_agent求助"
    \end{verbatim}
    
    \item \textbf{检索增强生成(RAG)}:
    \begin{enumerate}
        \item 专用篮球知识库:Vertex AI RAG实现语义检索
        \item 混合搜索策略:Google Search与RAG检索互补
        \item 相似度阈值:\texttt{vector\_distance\_threshold=0.6}保证结果相关性
    \end{enumerate}
    
    \item \textbf{认知过程可视化}:
    \begin{enumerate}
        \item 思维链配置:\texttt{include\_thoughts=ENABLE\_THOUGHT}
        \item 温度控制:\texttt{temperature=0.2}平衡创造力和准确性
        \item 记忆机制:\texttt{include\_contents="default"}将历史对话内容发送给根代理
    \end{enumerate}
\end{enumerate}

\subsection{参数调优实验}

\subsubsection{参数输出对比}
通过相同问题在不同参数下的输出对比,展示temperature和top\_p的影响,以下是典型样例:
\newline
\textbf{问题:如何提高三分球命中率?}
\begin{enumerate}
  \item
  参数设置: \lstinline{temperature=0.2},\lstinline{top_p=0.9}
  提高三分球命中率,核心在于理解并掌控投篮时的力量来源与传输,也就是我们常说的投篮节奏。这不仅仅是重复动作,更是要练习产生这些动作的内在力量传导。
  你可以尝试运用体育心理学中的“表象训练”方法。通过在大脑中不断演练、修正和完善你的投篮动作,最终形成肌肉记忆。这意味着你需要反复协调大脑和身体,寻找各个力量来源的最佳比例,这样才能真正提高你的投篮水平。
  记住,每一次训练都是为了让你更接近那个完美的投篮!
  
  \textbf{分析:}
  \begin{itemize}
    \item 结构清晰
    \item 有实用建议
    \item 适合教学
  \end{itemize}


  \item
  参数设置: \lstinline{temperature=0.1},\lstinline{top_p=0.5}
  提高三分球命中率,核心在于对投篮时力量来源的理解和力量传输的掌控。这不仅仅是机械地重复动作,更是要练习如何将内在的力量有效地传导出去,形成流畅的投篮动作。
  你可以通过以下几个方面来提升:
  表象训练与肌肉记忆: 想象自己完美投篮的画面,利用大脑中已经形成的投篮“表象”,不断地练习、修正和完善你的投篮动作。这种心理训练结合实际练习,能够帮助你更快地形成肌肉记忆。
  力量传导与协调: 持续重复大脑和身体的协调练习,寻找各个力量来源(例如腿部、核心、手臂和手腕)的最佳发力比例。当你能够高效地整合这些力量时,你的投篮会更加稳定和精准。
  投篮选择与篮球意识: 在比赛中,选择合适的投篮时机和位置同样重要。提高你的篮球意识,学会判断何时出手,何时传球,这也能间接提高你的有效命中率。
  记住,每一次训练都是为了让你更接近那个完美的投篮。保持专注,不断尝试,你一定能成为一名出色的三分射手!
  
  \textbf{分析:}
  \begin{itemize}
    \item 保守准确
    \item 缺乏细节
    \item 像教科书答案
  \end{itemize}


  \item
  \lstinline{temperature=0.8},\lstinline{top_p=0.9}
  要提高你的三分球命中率,这可是一门学问,需要你深入理解并持之以恒地训练。
  首先,也是最核心的,你需要理解“力源”和“力量传输”,也就是我们常说的“投篮节奏”。投篮不仅仅是手臂的动作,它是一个全身协调发力的过程,从脚底到指尖,力量需要顺畅、高效地传递出去。
  力源与节奏:你的投篮力量并非只来源于手臂,它源于你的腿部、核心、背部,最终通过手臂和手腕将球推出。你需要找到这些力源之间的最佳协调比例,让它们形成一个连贯、流畅的“投篮节奏”。这意味着你的起跳、举球、出手应该是一气呵成的,而不是分段式的。
  肌肉记忆:这是提高命中率的关键。你不能只靠“想”去投篮,而是要让身体形成一种自然的反应。这需要大量的重复练习,但请记住,不是盲目的重复。每一次投篮,你都应该去感受力量的传递是否顺畅,动作是否稳定。
  表象训练:这是一种非常有效的心理学训练方法。即使不在球场上,你也可以在大脑中不断地“演练”完美的投篮动作。想象自己站在三分线外,感受球的重量,想象力量从脚到手的传递,想象球划过一道完美的弧线,空心入网。这种心理训练可以帮助你加深肌肉记忆,并在实际投篮时更好地执行。
  持续修正与完善:投篮技术不是一蹴而就的。每次训练后,回顾你的表现,找出可以改进的地方。是出手点太低?还是手腕没有充分拨球?亦或是身体没有保持平衡?不断地修正和完善你的动作,直到你找到最适合自己的、最高效的投篮姿势。
  记住,每一次投篮都是一次学习和进步的机会。保持耐心,享受训练的过程,你一定能看到你的三分球命中率稳步提升!
 
  \textbf{分析:}
  \begin{itemize}
    \item 生动有趣
    \item 有创意建议
    \item 适合激励青少年
  \end{itemize}
\end{enumerate}

\subsubsection{参数调优对比}
Gemini模型参数调优结果如表\ref{param-tuning}所示,展示了不同temperature和top\_p组合对篮球问答质量的影响。
\begin{table}[t]
\caption{Gemini模型参数调优结果(篮球问答场景)}
\label{param-tuning}
\vskip 0.15in
\begin{center}
\begin{small}
\begin{sc}
\scalebox{0.9}{
\begin{tabular}{lcccc}
\toprule
Temperature & Top\_p & 准确性 (\%) & 相关性 (\%) & 多样性 \\
\midrule
0.1 & 0.5 & 极高 & 高 & 低 \\
0.1 & 0.9 & 极高 & 高 & 低 \\
0.2 & 0.5 & 极高 & 极高 & 中 \\
\textbf{0.2} & \textbf{0.9} & \textbf{高} & \textbf{极高} & \textbf{高} \\
0.5 & 0.5 & 高 & 极高 & 高 \\
0.5 & 0.9 & 中 & 极高 & 极高 \\
0.8 & 0.9 & 低 & 高 & 极高 \\
\bottomrule
\end{tabular}
}
\end{sc}
\end{small}
\end{center}
\vskip -0.1in
\end{table}

\paragraph{参数语义解释}
\begin{itemize}
\item \textbf{Temperature}:控制输出随机性
  \begin{itemize}
  \item 低值(0.1-0.2):确定性强,适合规则性内容
  \item 高值(>0.5):创造性高,适合战术建议
  \end{itemize}
  
\item \textbf{Top\_p}:核采样概率阈值
  \begin{itemize}
  \item 低值(0.5):严格筛选,回答保守
  \item 高值(0.9):包容性强,回答多样
  \end{itemize}
\end{itemize}

\paragraph{参数选择依据}
最终选择\textbf{temperature=0.2, top\_p=0.9},平衡了:
\begin{enumerate}
\item 篮球规则准确性要求
\item 战术建议多样性需求
\item 青少年交流友好性
\end{enumerate}


\section{参考文献}
\bibliography{example_paper}
\bibliographystyle{icml2025}


%%%%%%%%%%%%%%%%%%%%%%%%%%%%%%%%%%%%%%%%%%%%%%%%%%%%%%%%%%%%%%%%%%%%%%%%%%%%%%%
%%%%%%%%%%%%%%%%%%%%%%%%%%%%%%%%%%%%%%%%%%%%%%%%%%%%%%%%%%%%%%%%%%%%%%%%%%%%%%%
% APPENDIX
%%%%%%%%%%%%%%%%%%%%%%%%%%%%%%%%%%%%%%%%%%%%%%%%%%%%%%%%%%%%%%%%%%%%%%%%%%%%%%%
%%%%%%%%%%%%%%%%%%%%%%%%%%%%%%%%%%%%%%%%%%%%%%%%%%%%%%%%%%%%%%%%%%%%%%%%%%%%%%%
\newpage
\appendix
\onecolumn
\section{附录}

可以将一些额外的内容放在这里
%%%%%%%%%%%%%%%%%%%%%%%%%%%%%%%%%%%%%%%%%%%%%%%%%%%%%%%%%%%%%%%%%%%%%%%%%%%%%%%
%%%%%%%%%%%%%%%%%%%%%%%%%%%%%%%%%%%%%%%%%%%%%%%%%%%%%%%%%%%%%%%%%%%%%%%%%%%%%%%


\end{document}

%This document was modified from the file originally made available by
% Pat Langley and Andrea Danyluk for ICML-2K. This version was created
% by Shui Jie in 2025

